\documentclass{beamer}
%\documentclass[aspectratio=169]{beamer} % 使用 16:9 宽屏比例
\usepackage{ctex, hyperref}
\usepackage[T1]{fontenc}

% other packages
\usepackage{latexsym,amsmath,xcolor,multicol,booktabs,calligra}
\usepackage{graphicx,pstricks,listings,stackengine}

\author{Mark Robinson}
\title{Temporal-aware Bayesian inference for optimizing experimental returns}
\subtitle{STATS 209 - Project Presentation}
\institute{Stanford University}
\date{2025 12 03}
\usepackage{CNU}
\usepackage{natbib}

\bibliographystyle{agsm}
\setcitestyle{citesep={;}, aysep={,}}

%[<+-| alert@+>] 

% defs
\def\cmd#1{\texttt{\color{red}\footnotesize $\backslash$#1}}
\def\env#1{\texttt{\color{blue}\footnotesize #1}}
\definecolor{deepblue}{rgb}{0,0,0.5}
\definecolor{deepred}{rgb}{0.6,0,0}
\definecolor{deepgreen}{rgb}{0,0.5,0}
\definecolor{halfgray}{gray}{0.55}

\lstset{
    basicstyle=\ttfamily\small,
    keywordstyle=\bfseries\color{deepblue},
    emphstyle=\ttfamily\color{deepred},    % Custom highlighting style
    stringstyle=\color{deepgreen},
    numbers=left,
    numberstyle=\small\color{halfgray},
    rulesepcolor=\color{red!20!green!20!blue!20},
    frame=shadowbox,
}


\renewcommand{\figurename}{Figure}
\renewcommand{\tablename}{Table}
\setbeamerfont{caption}{family=\rmfamily}
\setbeamerfont{caption name}{family=\rmfamily}
\renewcommand{\refname}{References}
\renewcommand{\bibname}{References} % sometimes used by Beamer
\setbeamerfont{block body}{family=\kaishu}
\setbeamerfont{block title}{family=\heiti}
\setbeamerfont{bibliography item}{family=\rmfamily}
\setbeamerfont{bibliography entry author}{family=\rmfamily}
\setbeamerfont{bibliography entry title}{family=\rmfamily}
\setbeamerfont{bibliography entry location}{family=\rmfamily}
\setbeamerfont{bibliography entry note}{family=\rmfamily}
\setbeamertemplate{bibliography item}[text]

\begin{document}

\kaishu
\begin{frame}
    \titlepage
\end{frame}

\begin{frame}
    \tableofcontents[sectionstyle=show,subsectionstyle=show/shaded/hide,subsubsectionstyle=show/shaded/hide]
\end{frame}


\section{Introduction}

\begin{frame}{Background \& Motivation}
    \begin{itemize}
        \item Representative problem: Experimentation program design in the technology industry
        \item Competing approaches to program design:
            \begin{itemize}
                \item Traditional frequentist approach: searching for statistically significant results
                \item Bayesian approach: directly optimizing for business metrics (e.g. user satisfaction; click-through rate)
            \end{itemize}
        \item Potential gains from improvements?
            \begin{itemize}
            \item Faster and more accurate testing at lower cost             \begin{itemize}
                \item \textit{Preview}: This is achievable!
            \end{itemize}
            \end{itemize}
    \end{itemize}
\end{frame}

\section{Model}

\begin{frame}{Model set-up (outline)}
Model for experimental program design (e.g. \cite{azevedo2020b}, \cite{sudijono2025optimizing}) -- 
    \begin{itemize}
        \item Set-up: 
        \begin{itemize}
            \item Organization generates a set of ideas over time; can choose which ideas to implement.
            \item Ideas have some true (unobserved) value, organization would only like to implement good ideas. 
            \item Organization can carry out testing to learn about the true value of ideas, but faces limited testing resources.
        \end{itemize}
        \item Organization problem: Choose --  
        \begin{enumerate}
            \item\textbf{Testing procedure}: Which ideas to test, and how many resources to allocate to each test
            \item \textbf{Decision rule}: Given testing results, which ideas to `shelve' and which to `ship'
        \end{enumerate}
    \end{itemize}
\end{frame}

\begin{frame}{Model set-up (notation)}
    \begin{itemize}
        \item Set-up: 
        \begin{itemize}
            \item Discrete time periods $t$: $1, \dots, \infty$ 
            \item New ideas generated each period $i$: $1, \dots, I_t$
            \item Testing resources each period $n$: $1, \dots, N_t$
        \end{itemize}
        \item Ideas and testing:
        \begin{itemize}
            \item True idea values $\Delta_i$; prior over idea values: $\Delta_i \sim _{iid} (\mu,\tau^2)$.
            \item Signal for idea $i$ generated according to $\hat \Delta_i \sim (\Delta_i,\frac{\sigma^2}{N_i})$, where $N_i$ is the number of units allocated for testing for unit $i$, and $\sigma^2$ is some baseline level of noise.
        \end{itemize}
        \item Payoff:
        \begin{itemize}
            \item Organization payoff each period equal to: 
        \end{itemize}
            $$U_t = \sum_{\{i\in S_t\}} u(\Delta_i) = \sum_{\{i\in S_t\}} \Delta_i$$
    \end{itemize}
\end{frame}

\begin{frame}{Extending decisions to be `temporal-aware'}
    \begin{itemize}
    \item \textcolor{red}{New} organization problem: Choose --
        \begin{enumerate}
            \item\textbf{Testing procedure}: Which ideas to test, and how many resources to allocate to each test
            \item \textbf{Decision rule}: Given testing results \textcolor{red}{across all previous periods}, which ideas to `shelve', which to `ship' \textcolor{red}{and which to continue testing}
        \end{enumerate}
    \item Formulating the new problem:
        \begin{itemize}
            \item Let $I^c_t$ denote cumulative ideas up to period $t$, and similar for $\Delta^c_{i,t}, n^c_{i,t}$. Let $\gamma$ be a discount factor over the future. Then total expected reward from shipping ideas with positive posterior mean is:
        \end{itemize}
    $$\textcolor{red}{\sum_{t=1}^\infty \gamma^t }\sum_{i=1}^{I\textcolor{red}{^c_t}} \mathbb{E}[\mathbb{E}[u(\Delta_{i,\textcolor{red}{t}})|{\hat\Delta^\textcolor{red}{c}}_{i,\textcolor{red}{t}}; {n^\textcolor{red}{c}}_{i,\textcolor{red}{t}}] \textcolor{red}{1(n_{i,t}^c)} ]$$
        \begin{itemize}
            where $1(n_{i,t}^c)$ is an indicator of whether the inner expectation is larger than the `ship' threshold.
        \end{itemize}
    \end{itemize}
\end{frame}

\begin{frame}{Optimal organization decision-making}
The optimal organization decisions will cover the following:
    \begin{enumerate}
        \item Decision rule thresholds: Optimal shelve/continue testing/ship thresholds ($\alpha; \beta)$ over $\tilde \Delta_i$ for an idea, dependent on how times the idea has previously been tested: $\tilde\Delta_i < \alpha \implies$ shelve; $\tilde \Delta_i > \beta \implies$ ship; $\tilde \Delta_i \in [\alpha,\beta] \implies$ continue testing
         \begin{itemize}
            \item Optimal thresholds determine $\beta$ by $U_t(\beta) = \beta$; and $\alpha$ by $U_t(\alpha) = U_0$, where $U_0$ is the expected return from an untested idea.
        \end{itemize}
        \item Number of ideas to test each period $(k)$
         \begin{itemize}
            \item Assumption 1: Test every idea assigned to `continue testing'
            \item Assumption 2: Test the same number of ideas overall each period ($k_t = k_{t+1}$)
            \item Assumption 3: Allocate resources equally between tested ideas each period ($n_{i,t} = \frac{N_t}{k}$). Motivation -- \cite{sudijono2025optimizing}.
        \end{itemize}
    \end{enumerate}    
\end{frame}

\begin{frame}{Solution approach}
Constructing a solution -- 
    \begin{itemize}
        \item Solve using numerical methods. For simplicity:
        \begin{itemize}
            \item Normal prior distribution over ideas derived from Netflix experimentation data \citep{sudijono2025optimizing} [Work in progress, placeholder values used for now]
            \item Risk-neutral utility function $U(x) = x$
        \end{itemize}
        \item Solving proceeds iteratively as follows:
        \begin{enumerate}
            \item Deriving decision rule thresholds:
             \begin{itemize}
                \item Given prior and parameters, $k$, and a guess/estimate for $U_0$, identify optimal shelve/continue/ship thresholds
                \item Given optimal thresholds, derive $U_0$. 
                \item Repeat until convergence.
            \end{itemize}
            \item Deriving optimal number of ideas to test:
             \begin{itemize}
                \item Given decision rule thresholds, calculate implied expected utility from each possible number of ideas. Choose $n^*$ to maximize expected utility.
            \end{itemize}
        \end{enumerate}    
            
        \end{itemize}
\end{frame}

\section{Results}

\begin{frame}{Optimal program design (decision thresholds)}
Optimal decision thresholds at $\mu = -1.0$; $\tau = 1.0$; $\sigma^2 = 2.0$; $N = 1000$; $k=1$ and $\gamma = 0.99$
\begin{figure}
        \centering
        \includegraphics[width=0.8\linewidth]{Thresholds}
\end{figure}    
Intuition -- early: prioritize further testing; later: only re-test very uncertain cases!
\end{frame}

\begin{frame}{Optimal program design (action probabilities)}
Probabilities of each action implied by decision thresholds:
\begin{figure}
        \centering
        \includegraphics[width=0.8\linewidth]{ActionP.png}
\end{figure}
Intuition? Given most ideas are bad, immediately discard most unless strong evidence in favor.
\end{frame}

\begin{frame}{Optimal program design (number of ideas)}
\begin{itemize}
        \item \textbf{Naive return}: Expected payoff if decision had to be made after one period. Equal to posterior mean 
        \item \textbf{Temporal-aware return}: Expected payoff from following decision rule. Allows for higher return on average within `continue' range. 
\end{itemize}
\begin{figure}
        \centering
        \includegraphics[width=0.8\linewidth]{Figure_1.png}
\end{figure}    
\end{frame}

\begin{frame}{Optimal program design (payoff)}
Expected payoff changing with number of ideas selected to test each period, given optimal decision thresholds.
\begin{figure}
        \centering
        \includegraphics[width=0.8\linewidth]{Ideas.png}
\end{figure}   
Intuition? Trade-off between increasing the number of ideas tested each period and decreasing testing power due to finite resources.
\end{frame}

\begin{frame}{Defining a baseline strategy for comparison}
\begin{itemize}
    \item Compare `temporal-aware' results to a restricted strategy -- without the option to continue testing.
    \item Chooses $t^*$ periods over which to accumulate signals and $k^*$ ideas. `Ship' / `shelve' each tested unit after $t^*$ periods:
\end{itemize} 
\begin{figure}
        \centering
        \includegraphics[width=0.8\linewidth]{Naive.png}
\end{figure}   
\end{frame}

\begin{frame}{Strategy comparison}
Use average per-period naive strategy and temporal-aware payoffs to compare the two approaches.
With these parameters, `temporal-awareness' has:
\begin{itemize}
    \item $\approx 70x$ faster testing per idea
    \item $\approx 20x$ higher per-period ideas tested 
    \item $\approx 5x$ higher per-period expected return
\end{itemize} 
Results are robust to different parameters. 
Temporal awareness exhibits even higher relative performance with lower $\mu$ and $\sigma^2$; and higher $\gamma$.
\end{frame}

\section{Outlook}

\begin{frame}{Extensions}
Various modifications / extensions are possible which lead to a richer model:
    \begin{itemize}
        \item Cost of ideation (increases `continue' rate!)
        \item Risk-aversion (increases `continue' rate!)
        \item Cost of implementation (?)
        \item Alternative distributions (?)
    \end{itemize}
As well as considering these, my next steps are:
\begin{itemize}
    \item Repeat analysis using real-world data from Netflix
    \item Consider links to wider literature including sequential frequentist testing
\end{itemize}
\end{frame}

\begin{frame}{Thanks}
    \begin{center}
        {\Large{Thanks for listening!}}
    \end{center}
    \begin{center}
        {Even if you are not assigned to me for peer feedback, any thoughts/comments still welcome at mrobin10@stanford.edu (or in person).}
    \end{center}
\end{frame}

\begin{frame}
References:
   \bibliography{ref}
\end{frame}


\end{document}
